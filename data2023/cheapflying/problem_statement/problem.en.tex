\problemname{Cheap Flying}

\illustration{0.25}{benelux.pdf}{
    The unpredictability of the jury members would be even worse, if they would all fly from their own universities,
    instead of from the headquarters of BAPC. \\
    CC BY-SA 4.0 by Shaund on \href{https://commons.wikimedia.org/wiki/File:Benelux-map.svg}{Wikimedia Commons}, modified
}

% optionally define variables/limits for this problem
\newcommand{\maxa}{10^6}

You are part of the organization of BAPC, and you are in charge of flight operations.
From time to time, the jury members of BAPC
need to fly from the remote headquarters of BAPC to the current location of the operation that BAPC performs,
which is currently Amsterdam.
Luckily there is an airline that serves the route from BAPC's headquarters to Amsterdam.
The BAPC organization has a contract with that airline that ensures that you pay the same fixed amount for each flight.
If your judges need to fly the route often, these costs can add up and become really high.
To avoid this, you figured out that you could also simply buy your own aircraft.
Once you own your own aircraft, you can either fly with this new shiny
equipment which costs some fixed price per flight for fuel and the like, or
alternatively you could still fly with the airline.

The problem now is that you have no idea what the judges are doing!
They are so incredibly unpredictable and always only do random and very
complicated things.
So random, that you cannot decide beforehand what you should do.
So you need to make your decision on the fly, even though this may prevent you
from making the cost-optimal decision up-front.
Still, you do not want to be too loose with your spending:
you set yourself the constraint that you spend at most twice as much as you
would have if you exactly knew how many flights the judges would make in advance.

\begin{Interaction}
    This is an interactive problem.
    Your submission will be run against an \emph{interactor},
    which reads from the standard output of your submission
    and writes to the standard input of your submission.
    This interaction needs to follow a specific protocol:

    The interactor first sends
    one line with three integers $a$, $b$, and $c$ ($0\leq a,b,c\leq \maxa$),
    the cost of one flight with the airline,
    the one-time cost of buying an aircraft, and
    the cost for each flight using your own aircraft.

    Then, the interactor sends either the input ``\texttt{flight}'' or the input ``\texttt{end}''.
    For every input ``\texttt{flight}'', you need to provide either
    ``\texttt{airline}'' if you want fly with the airline for cost $a$,
    ``\texttt{buy}'' to buy an aircraft and use it for a single flight for cost $b+c$, or
    ``\texttt{self}'' if you want to use your own aircraft for cost $c$.
    You can only use the last option if you have bought an aircraft before, and you can only buy an aircraft once.

    The interaction ends when you receive the input ``\texttt{end}'', which is after at most $10^4$ rounds.

    The interactor is adaptive, and may send fewer or more flights based on your output.

    Make sure you flush the buffer after each write.

    A testing tool is provided to help you develop your solution.
\end{Interaction}
