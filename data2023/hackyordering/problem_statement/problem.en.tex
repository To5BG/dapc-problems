\problemname{Hacky Ordering}

\illustration{0.2}{illustration.pdf}{
    If only you were allowed to use this universal sorting function\dots
}

% optionally define variables/limits for this problem
\newcommand{\maxn}{10^5}
\newcommand{\maxs}{10^5}

You have been asked to sort! Again! For the bazillionth time!
Not even numbers, but strings! Ugh!
Do people still not have this in their standard library?
Why do you even need to learn this?
Who even uses a language without \texttt{sort} function\textinterrobang

Clearly, you have not been paying attention in class for such a stupid ubiquitous function,
but now you have been asked to implement it!
Without calling \texttt{sort}! But you just cannot!

But wait!
You have a better approach:
what if you just assume that the list is sorted already?
The order of the characters in the alphabet is arbitrary anyway\ldots{}
So, instead of sorting the list,
you want to determine whether there exists some order of the characters of the alphabet
such that the list of strings is sorted according to this order.

Note that when a string is a prefix of some longer string, the shorter string
should be sorted before the longer string.

\begin{Input}
    The input consists of:
    \begin{itemize}
        \item One line with an integer $n$ ($1\leq n\leq \maxn$), the number of strings.
        \item $n$ lines, each with a string.
    \end{itemize}
    The strings only consist of English lowercase letters (\texttt{a-z}). \\
    The total number of characters in the $n$ strings is at most $\maxs$. \\
    The strings are not necessarily distinct.
\end{Input}

\begin{Output}
    If it is impossible to determine an order of the alphabet, output ``\texttt{impossible}''. \\
    If it is possible, output a permutation of the 26 letters of the English alphabet
    according to which the strings are sorted.

    If there are multiple valid solutions, you may output any one of them.
\end{Output}
